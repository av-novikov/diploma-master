\begin{thebibliography}{99}
\addcontentsline{toc}{section}{Список использованных источников}

% Механика и термодинамика
\bibitem{charniy} Чарный И.А. Подземная гидрогазодинамика. -- М.--Ижевск: НИЦ <<Регулярная и хаотическая динамика>>, Институт компьютерных исследований, 2006. -- 436 стр.
\bibitem{basniev} Басниев К.С., Кочина И.Н., Максимов В.М. Подземная гидромеханика: Учебник для вузов. -- М.: Недра, 1993. 416 с.: ил
\bibitem{multiphase} Розенберг М.Д., Кундин С.А. Многофазная многокомпонентная фильтрация при добыче нефти и газа. М., <<Недра>>, 1976. 335 c.
\bibitem{fortov} Кондауров В.И., Фортов В.Е. Основы термомеханики конденсированной среды. -- М.: Издательство МФТИ, 2002. -- 336 с.
\bibitem{kondaurov} Кондауров В.И. Механика и термодинамика насыщенной пористой среды: Учебное пособие. -- М.: МФТИ, 2007. -- 310 с.
\bibitem{checkalyuk} Чекалюк Э.Б. Термодинамика нефтяного пласта. -- М.: Недра, 1965. - 238 с.
\bibitem{nigmatulin} Нигматулин Р.И. Динамика многофазных сред. Ч. I. М.: Наука. Гл. ред. физ-мат. лит. 1987 -- 464 с.

% Вычислительные методы
\bibitem{petrov} Петров И.Б., Лобанов А.И. Лекции по вычислительной математике: Учебное пособие -- М.: Интернет-Университет Информационных Технологий; БИНОМ.Лаборатория знаний, 2013.-523 с.: ил., табл. -- (Серия <<Основы информационных технологий>>)
\bibitem{kanevskaya} Каневская Р.Д. Математическое моделирование гидродинамических процессов разработки месторождений углеводородов. -- Москва-Ижевск: Институт компьютерных исследований, 2003, 128 стр.
\bibitem{chen} Chen Zhangxin, Guanren Huan, and Yuanle Ma. Computational Methods for Multiphase Flows in Porous Media. Philadelphia: Society for Industrial and Applied Mathematics, 2006.
\bibitem{leveque} LeVeque R.J. Finite volume methods for Hyperbolic problems. Cambridge University Press, 2002.
\bibitem{keldysh} Марченко Н.А. [и др.] / Иерархия явно-неявных разностных схем для решения задачи многофазной фильтрации // Препринты ИПМ им. Келдыша. 2008. № 97. 17 с. URL: \url{http://library.keldysh.ru/preprint.asp?id=2008-97}

% Рамазанов
\bibitem{ramazanov_diss} Рамазанов А.Ш. Теоретические основы термогидродинамических методов исследования нефтяных пластов. Автореф. дис. докт. техн. наук. - Уфа, 2004.
\bibitem{ramazanov_main} Рамазанов А.Ш., Паршин А.В. Температурное поле в нефте-водонасыщенном пласте с учётом разгазирования нефти // Электронный научный журнал «Нефтегазовое дело». 2006. №1. URL: \url{http://ogbus.ru/authors/Ramazanov/Ramazanov_1.pdf}
\bibitem{ramazanov_spe} Ramazanov A.Sh., Valiullin R.A., Sadretdinov A.A., Shako V.V., Pimenov V.P., Fedorov V.N., Belov K.V. Thermal Modeling for Characterization of Near Wellbore Zone and Zonal Allocation. SPE 136256, Moscow: SPE Russian Oil and Gas Conference and Exhibition, 2010.
\bibitem{ramazanov_spe1} Валиуллин Р.А., Рамазанов А.Ш., Хабиров Т.Р., Садретдинов А.А., Шако В.В., Сидорова М.В., Котляр Л.А., Федоров В.Н., Салимгареева Э.М. Интерпретация термогидродинамических  исследований при испытании скважины на основе численного симулятора. SPE-176589-RU, Российская нефтегазовая техническая конференция SPE, 26-28 октября, 2015, Москва, Россия.

% Нефтянка
\bibitem{kappa} Оливье Узе, Дидье Витура, Оле Фьярэ. Анализ динамических потоков. КАППА выпуск v4.10.01 - Октябрь 2008.

% Посвянский и др.
\bibitem{posv} Posvyanskii D.V., Gaidukov L.A., Tukhvatullina R.R. Estimating Bottom Hole Damage Zone Parameters Based on Mathematical Model of Thermo-hydrodynamic Processes // ECMOR XIV. - 2014.

\end{thebibliography}
	
