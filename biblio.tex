\begin{thebibliography}{99}
\addcontentsline{toc}{section}{Список использованных источников}

% Механика и термодинамика
\bibitem{charniy} Чарный И.А. Подземная гидрогазодинамика. -- М.--Ижевск: НИЦ <<Регулярная и хаотическая динамика>>, Институт компьютерных исследований, 2006. -- 436 стр.
\bibitem{basniev} Басниев К.С., Кочина И.Н., Максимов В.М. Подземная гидромеханика: Учебник для вузов. -- М.: Недра, 1993. 416 с.: ил
\bibitem{multiphase} Розенберг М.Д., Кундин С.А. Многофазная многокомпонентная фильтрация при добыче нефти и газа. М., <<Недра>>, 1976. 335 c.
\bibitem{fortov} Кондауров В.И., Фортов В.Е. Основы термомеханики конденсированной среды. -- М.: Издательство МФТИ, 2002. -- 336 с.
\bibitem{kondaurov} Кондауров В.И. Механика и термодинамика насыщенной пористой среды: Учебное пособие. -- М.: МФТИ, 2007. -- 310 с.
\bibitem{checkalyuk} Чекалюк Э.Б. Термодинамика нефтяного пласта. -- М.: Недра, 1965. - 238 с.
\bibitem{nigmatulin} Нигматулин Р.И. Динамика многофазных сред. Ч. I. М.: Наука. Гл. ред. физ-мат. лит. 1987 -- 464 с.
\bibitem{alishaev} Алишаев М.Г., Розенберг М.Д., Теслюк Е.В. Неизотермическая фильтрация при разработке нефтяных месторождений. / Под ред. Г.Г. Вахитова -- М.: Недра, 1985. 271 с.
\bibitem{masket} Маскет М. Течение однородных жидкостей в пористой среде. -- Москва-Ижевск: Институт компьютерных исследований, 2004, 628 стр.

% Математика
\bibitem{vladimirov} Владимиров В.С. Уравнения математической физики. -- изд. 4-е -- М.: Наука. Главная редакция физико-математической литературы, 1981. -- 512 с.

% Вычислительные методы
\bibitem{petrov} Петров И.Б., Лобанов А.И. Лекции по вычислительной математике: Учебное пособие -- М.: Интернет-Университет Информационных Технологий; БИНОМ.Лаборатория знаний, 2013.-523 с.: ил., табл. -- (Серия <<Основы информационных технологий>>)
\bibitem{kanevskaya} Каневская Р.Д. Математическое моделирование гидродинамических процессов разработки месторождений углеводородов. -- Москва-Ижевск: Институт компьютерных исследований, 2003, 128 стр.
\bibitem{aziz} Азиз Х., Сеттари Э. Математическое моделированиие пластовых систем. -- Москва-Ижевск: Институт компьютерных исследований, 2004, 416 стр.
\bibitem{chen} Chen Zhangxin, Guanren Huan, and Yuanle Ma. Computational Methods for Multiphase Flows in Porous Media. Philadelphia: Society for Industrial and Applied Mathematics, 2006.
\bibitem{leveque} LeVeque R.J. Finite volume methods for Hyperbolic problems. Cambridge University Press, 2002.
\bibitem{keldysh} Марченко Н.А. [и др.] / Иерархия явно-неявных разностных схем для решения задачи многофазной фильтрации // Препринты ИПМ им. Келдыша. 2008. № 97. 17 с. URL: \url{http://library.keldysh.ru/preprint.asp?id=2008-97}

% Рамазанов
\bibitem{ramazanov_diss} Рамазанов А.Ш. Теоретические основы термогидродинамических методов исследования нефтяных пластов. Автореф. дис. докт. техн. наук. - Уфа, 2004.
\bibitem{ramazanov_main} Рамазанов А.Ш., Паршин А.В. Температурное поле в нефте-водонасыщенном пласте с учётом разгазирования нефти // Электронный научный журнал «Нефтегазовое дело». 2006. №1. URL: \url{http://ogbus.ru/authors/Ramazanov/Ramazanov_1.pdf}
\bibitem{ramazanov_spe} Ramazanov A.Sh., Valiullin R.A., Sadretdinov A.A., Shako V.V., Pimenov V.P., Fedorov V.N., Belov K.V. Thermal Modeling for Characterization of Near Wellbore Zone and Zonal Allocation. SPE 136256, Moscow: SPE Russian Oil and Gas Conference and Exhibition, 2010.
\bibitem{ramazanov_spe1} Валиуллин Р.А., Рамазанов А.Ш., Хабиров Т.Р., Садретдинов А.А., Шако В.В., Сидорова М.В., Котляр Л.А., Федоров В.Н., Салимгареева Э.М. Интерпретация термогидродинамических  исследований при испытании скважины на основе численного симулятора. SPE-176589-RU, Российская нефтегазовая техническая конференция SPE, 26-28 октября, 2015, Москва, Россия.
\bibitem{ramazanov_old1} Filippov A.I, Ramazanov A.Sh. Calculation of the Thermal Field of the Throttle Element of Apparatus for Studying the Joule-Thomson Effect. - Journal of Engineering Physics, 1980, Vol.38, Issue 2, pp. 203-207.
\bibitem{ramazanov_old2} Ramazanov A.Sh., Filippov A.I. Temperature Fields in the Case of Unsteady Flow in Porous Media. – Fluid Dynamics, 1983,  vol.18, N4, pp.646-649. 
\bibitem{valiullin} Валиуллин Р.А. Термические методы диагностики нефтяных пластов и поиска скважин /   Автореф. дис. докт. техн. наук. - Тверь, 1996.

% Нефтянка
\bibitem{kappa} Оливье Узе, Дидье Витура, Оле Фьярэ. Анализ динамических потоков. КАППА выпуск v4.10.01 - Октябрь 2008.
\bibitem{civan} Civan, Faruk. Reservoir formation damage : fundamentals, modeling, assessment, and mitigation. // Gulf Professional Publishing. - 2007. - P. 1135.
\bibitem{tariq} Karakas M. and Tariq S. «Semi-Analytical Productivity Models for Perforated Completions», paper SPE 18271, 1988.
\bibitem{erlauger} Р. Эрлагер мл. Гидродинамические методы исследования скважин. -- Москва-Ижевск: Институт компьютерных исследований, 2006. -- 512 стр.
\bibitem{brusilovskiy} А.И. Брусиловский. Фазовые превращения при разработке месторождений нефти и газа. М.: Грааль, 2002, 575 с.
\bibitem{mukerdzhi} Мукерджи Х. Производительность скважин. Изд. 2-е доп. М., 2001. -- 184 с.
\bibitem{makarova} Макарова А.А. Моделирование динамики изменения фильтрационных и электрических свойств околоскважинной зоны с целью оценки её загрязнения. Дисс. канд. техн. наук. РГУНГ им. Губкина, Москва -- 2015.
\bibitem{math} Arfken, G.B. and Weber, H.J. 1995. \textit{Mathematical Methods for Physicists}, fourth edition. San Diego: Academic Press.
\bibitem{main} Radek Pecher, Breaking the Symmetry with the multi-Point Well Connection Method. SPE-173302-MS. SPE Reservoir Simulation Symposium, Houston, 2015.
\bibitem{diss} Radek Pecher, Boundary Element Simulation of Petroleum Reservoirs with Hydraulically Fractured Wells. PhD thesis. The Univercity of Calgary, 1999.
\bibitem{shara} Шарафутдинов Р.Ф. Многофронтовые фазовые переходы при неизотермической фильтрации газированной парафинистой нефти. // Журн. ''Прикладная механика и техническая физика'', Т. 42, № 2, 2001.

% Посвянский и др.
\bibitem{posv} Posvyanskii D.V., Gaidukov L.A., Tukhvatullina R.R. Estimating Bottom Hole Damage Zone Parameters Based on Mathematical Model of Thermo-hydrodynamic Processes. // ECMOR XIV. - 2014.
\bibitem{posv1} D.V. Posvyanskii, A.B. Starostin, V.S. Posvyanskii, E.S. Makarova, A.A. Vorobjev. An Application of Green Function Technique and Ewald's Algorithm for Well Test Analysis. // ECMOR XI. - 2008.
\bibitem{posv2}A. Korneev, A.V. Novikov, D.V. Posvyanskii, V.S. Posvyanskii. An Application of Green's Function Technique for Computing Well Inflow without Radial Flow Assumption. // ECMOR XV. -- 2016.
\bibitem{aziz_green} Wolfsteiner, C., Durlofsky, L. J. and Aziz, K.: “Calculation of Well Index for Nonconventional Wells on Arbitrary Grids”, Computational Geosciences, 7, 61-82, 2003
\bibitem{duru} Obinna Duru, Roland N. Horne. Modeling Reservoir Temperature Transients and Matching to Permanent Downhole Gauge Data for Reservoir Parameter Estimation. 2008 SPE Annual Technical Conference and Exhibition, Denver.

\end{thebibliography}
	
