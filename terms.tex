\section*{Обозначения}
\addcontentsline{toc}{section}{Обозначения}
\setcounter{subsection}{0}

	Здесь и далее будем обозначать индексом с заглавной буквой $A = \{F, G, S\}$ эффективные характеристики соответствующего континуума, со строчной $a = \{f, g, s\}$ -- истинные характеристики.

	Нижними индексами $\{w, e\}$ будем обозначать значение величины на забое скважины и контуре пласта соответствено.
	
	Верхний индекс ``\textit{0}'' означает равновесную составляющую физической величины, в то время как ``\textit{dis}'' -- диссипативную составляющую.

	Обозначения и операторы:
\begin{where}
	\item [\mathbb{E}^3] трёхмерное евклидово пространство,
	\item [\kappa_A] отсчётная конфигурация континуума $A$,
	\item [\chi(t)] текущая конфигурация системы,
	\item [\nabla_{\kappa}] градиент в переменных $\boldsymbol{X}$,
	\item [\nabla] градиент в переменных $\boldsymbol{x}$,
	\item [\otimes] тензорное умножение,
	\item [\boldsymbol{\varepsilon}] абсолютный антисимметричный тензор 3-го ранга Леви-Чивита,
	\item [\boldsymbol{n}] вектор единичной нормали к поверхности,
\end{where}

	Общие характеристики среды:
\begin{where}
	\item [\boldsymbol{X}_A] радиус-вектор материальной точки континуума $A$ в отсчётной конфигурации,
	\item [\boldsymbol{x}] радиус-вектор материальной точки в актуальной конфигурации,
	\item [\boldsymbol{v}] вектор скорости материальной точки континуума $A$,
	\item [\boldsymbol{w}_A] в зависимости от контекста: либо дифузионная скорость (относительно центра масс среды), либо скорость движения флюида $A$ относительно скелета $S$,
	\item [\boldsymbol{F}_A] градиент деформаций (дисторсия) среды $A$,
	\item [\phi_A] объёмная доля континуума $A$,
	\item [\phi] пористость среды,
	\item [S] насыщенность среды нефтью (жидкой фазой),
	\item [\rho] плотность массы,
	\item [q_A] объёмная интенсивность перехода массы вещества в континуум $A$,
\end{where}

	Характеристики напряжённого состояния:
\begin{where}
	\item [\boldsymbol{t}] вектор напряжений,
	\item [\boldsymbol{\sigma}] тензор напряжений Коши,
	\item [\boldsymbol{b}] плотность массовых сил,
	\item [\boldsymbol{b}^{int}] объёмная сила взаимодействия континуумов,
\end{where}

	Термические xарактеристики:
\begin{where}
	\item [\theta] температура среды,
	\item [r] плотность внешних источников,
	\item [r^{int}] скорогсть объёмного теплообмена между континумами,
	\item [h_A] поверхностный приток тепла,
	\item [\boldsymbol{q}] вектор теплового потока,
	\item [e] плотность полной энергии,
	\item [u] плотность внутренней энергии,
	\item [\eta] плотность энтропии,
	\item [\psi] плотность свободной энергии,
	\item [\delta] диссипация энергии,
\end{where}

