\section*{Обозначения}
\addcontentsline{toc}{section}{Обозначения}
\setcounter{subsection}{0}
	
	Индексы:
\begin{where}
	\item [A = \{F, G, S\}] индекс континуума,
\end{where}

	Обозначения и операторы:
\begin{where}
	\item [\mathbb{E}^3] трёхмерное евклидово пространство,
	\item [\kappa_A] отсчётная конфигурация континуума $A$,
	\item [\chi(t)] текущая конфигурация системы,
	\item [\nabla_{\kappa}] градиент в переменных $\boldsymbol{X}$,
	\item [\nabla] градиент в переменных $\boldsymbol{x}$,
	\item [\otimes] тензорное умножение,
	\item [\boldsymbol{\varepsilon}] абсолютный антисимметричный тензор 3-го ранга Леви-Чивита,
\end{where}

	Общие характеристики среды:
\begin{where}
	\item [\boldsymbol{X}_A] радиус-вектор материальной точки континуума $A$ в отсчётной конфигурации,
	\item [\boldsymbol{x}] радиус-вектор материальной точки в актуальной конфигурации,
	\item [\boldsymbol{v}_A] вектор скорости материальной точки континуума $A$,
	\item [\boldsymbol{w}_A] в зависимости от контекста: либо дифузионная скорость (относительно центра масс среды), либо скорость движения флюида $A$ относительно скелета $S$,
	\item [\rho_A] эффективная плотность массы континуума $A$,
	\item [\rho_a] истинная плотность массы континуума $a$,
	\item [\boldsymbol{F}_A] градиент деформаций (дисторсия) среды $A$,
\end{where}

	Характеристики напряжённого состояния:
\begin{where}
	\item [\boldsymbol{\sigma}_A] тензор эффективных (парциальных) напряжений Коши для континуума A,
	\item [\boldsymbol{\sigma}_a] тензор истинных напряжений Коши для континуума A,
\end{where}

	Термические xарактеристики:

