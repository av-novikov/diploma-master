\section*{Глава 1\\Основы механики и термодинамики насыщенных пористых сред}
\addcontentsline{toc}{section}{Глава 1. Основы механики и термодинамики насыщенных пористых сред}
\setcounter{section}{1}
\setcounter{subsection}{0}
\setcounter{equation}{0}

	Насыщенная пористая среда -- совокупность твёрдого деформируемого скелета и флюида, насыщающего этот скелет.
	Под \textit{флюидом} понимается смесь жидкостей и газов, способная перемещаться внутри порового пространства скелета.
	Для описания совместного движения скелета и флюида используется \textit{гипотеза суперпозиции континуумов}, которая предполагает что в каждой точке пространства находится и скелет, и флюид. 

	Флюид, в свою очередь, может быть многофазным и многокомпонентным. Здесь и далее будем отождествлять понятие \textit{компоненты} с химическим веществом, входящим в состав флюида. \textit{Фазой} будем называть термодинамически равновесное состояние вещества, качественно отличное от других равновесных состояний того же вещества. 

	Подробное изложение законов сохранения, построение определяющих соотношений для таких систем можно найти в \cite{basniev, fortov, kondaurov, checkalyuk}. Вопросы динамики многофазных сред описаны в \cite{nigmatulin}.
	
	Здесь и далее, для простоты, будем рассматривать флюид, состоящий из двух компонент, которые могут находится в двух фазах (жидкой и газообразной) -- \textit{бинарную смесь}.
	В данной главе будут представлены законы сохранения, основные определяющие соотношения такой системы. Как результат будут получены математические модели 	процессов массо- и теплопереноса в пористых средах.

\subsection{Кофигурации. Градиент деформации. Подходы Эйлера и Лагранжа описания движения сплошной среды.}
	\textit{Материальная точка} или \textit{элементарный объём} -- объём сплошной среды, пренебрежимо малый по сравнению с размерами рассматриваемой задачи, но, при том,   достаточный для того чтобы можно было проводить по нему осреднение. Дальнейшее рассмотрение будет проводится именно для таких объёмов.
	
	Обозначим $\kappa_A \in \mathbb{E}^3$ -- область, которую занимают частицы скелета (A=S) или флюида (A=F, G) в момент времени $t=0$. Область $\kappa_A$ в дальнейшем будем называть \textit{отсчётной (начальной) конфигурацией} скелета или флюида соответственно. 
	Область $\chi(t) \in \mathbb{E}^3$, занятую в момент времени $t > 0$ частицами скелета и флюида, назовём \textit{актуальной} или \textit{текущей конфигурацией}. Отображения $\kappa_A \to \chi(t)$ будем называть деформацией концинуума $A$.
	
	Здесь и далее предполагается, что области $\kappa_A$, $\chi(t)$ -- регулярны, отображения $\kappa_A \to \chi(t)$ -- кусочно-гомеоморфны и дифференцируемы. Тогда существуют взаимнооднозначные дифференцируемые связи:
\begin{equation}
	\label{motion_law}
	\boldsymbol{x} = \boldsymbol{x}(\boldsymbol{X}_A, t), \quad t > 0,\quad \boldsymbol{x} \in \chi(t), \quad \boldsymbol{X}_A \in \kappa_A,
\end{equation}
	которые называются \textit{законами движения} материальных точек скелета и флюида.
	
	Возмём дифференциал от \eqref{motion_law}:
\begin{align}
	\label{gradient}
	d\boldsymbol{x} = d\boldsymbol{X}_A \cdot \left(\nabla_{\kappa} \otimes \boldsymbol{x} \right) &= d\boldsymbol{X} \cdot \boldsymbol{F}_A^T = \boldsymbol{F}_A \cdot d\boldsymbol{X},\nonumber\\
	\boldsymbol{F}_A(\boldsymbol{X}, t) &= \left[\nabla_{\kappa} \otimes \boldsymbol{x}(\boldsymbol{X}_A, t)\right]^T,
\end{align}
	где $\boldsymbol{F}_A$ -- тензор второго ранга, называемый \textit{градиентом деформации (дисторсией)} континуума A.
	
	Для градиента деформаций $\textbf{F}_A$ справедлива \textit{теорема Коши о полярном разложении}, которая позволяет представить деформацию элемента $d\boldsymbol{X}_A$ как комбинацию растяжения (сжатия) и вращения как жесткого целого:
\begin{equation}
	\label{coushy_th}
	\boldsymbol{F}_A = \boldsymbol{R}_A \cdot \boldsymbol{U}_A = \boldsymbol{V}_A \cdot \boldsymbol{R}_A,
\end{equation}
	где $\boldsymbol{R}_A$ -- ортогональный тензор второго ранга, называемый \textit{тензором поворота},
	$\boldsymbol{U}_A, \boldsymbol{V}_A$ -- симметричные положительной определённые тензоры второго ранга, называемые \textit{правым и левым тензорами растяжения}. Разложение \eqref{coushy_th} единственно.
	
	Частной производной закона движения \eqref{motion_law} по времени является \textit{вектор скорости материальной точки}:
\begin{equation}
	\label{velocity}
	\boldsymbol{v}_A(\boldsymbol{X}_A, t) \equiv \dot{\boldsymbol{x}}(\boldsymbol{X}_A, t) = \left.\frac{\partial\boldsymbol{x}(\boldsymbol{X}, t)}{\partial t}\right|_{\boldsymbol{X}_A}.
\end{equation}
	Здесь и далее точкой будем обозначать \textit{материальную производную по времени} (при постоянном $\boldsymbol{X}_A$).
	
	Описание характристик материальной точки функциями от $\boldsymbol{X}_A$ носит название \textit{материального} или \textit{лагранжевого описания среды}, а радиус-вектор $\textbf{X}_A$ носит название \textit{материальной} или \textit{лагранжевой переменной}. Если же характеристики представляются функциями $\boldsymbol{x}$, то такой подход называется \textit{пространственным} или \textit{эйлеровым описанием среды}, переменная $\boldsymbol{x}$ -- \textit{пространственной} или \textit{эйлеровой переменной}.

\subsection{Тезоры деформаций. Уравнение совместности скоростей и деформаций.}
	Для того чтобы охарактеризовать деформации континуума вводятся специальные меры -- \textit{тензоры конечных деформаций}. Наиболее употребительными являются \textit{тензоры Коши-Грина} $\boldsymbol{E}_A$ \textit{и Альманзи} $\boldsymbol{A}_A$:
\begin{align}
	\label{strain_measures}
	\boldsymbol{E}_A &= \frac{1}{2}\left(\boldsymbol{F}_A^T \cdot \boldsymbol{F}_A - \boldsymbol{I}\right)\\
	\boldsymbol{A}_A &= \frac{1}{2}\left(\boldsymbol{I} - \boldsymbol{F}_A^{-1T} \cdot \boldsymbol{F}_A^{-1}\right).
\end{align}

	Представляя градиент деформаций \eqref{gradient} через вектор перемещений $\boldsymbol{u}_A = \boldsymbol{x}_A - \boldsymbol{X}_A$, подставляя в \eqref{strain_measures} и пренебрегая членами второго порядка малости, получим \textit{тензор малых деформаций} $\boldsymbol{e_A}$:
\begin{equation}
	\label{small_strains}
	\boldsymbol{e}_A = \frac{1}{2}\left(\left(\nabla \otimes \boldsymbol{u}\right) + \left(\nabla\otimes \boldsymbol{u}\right)^T\right),
\end{equation}
	где в принятых допущениях: $\nabla_{\kappa} \simeq \nabla$.

	Величины $\boldsymbol{F}_A(\boldsymbol{X}_A, t)$ и $\boldsymbol{v}_A(\boldsymbol{X}_A, t)$ являются первыми производными отображения $\kappa_A \to \chi(t)$. Предполагая отображение \eqref{motion_law} кусочно дважды непрерывно-дифференцируемым, получим соотношение:
\begin{equation}
	\label{velocities_strains}
	\dot{\boldsymbol{F}}_A = \left(\nabla_{\kappa} \otimes \boldsymbol{v}_A \right)^T,
\end{equation}
	называемое \textit{уравнением совместности скоростей и деформаций}.

\subsection{Пористость. Эффективная и истинная плотности. Закон сохранения масс.}

	Для описания доли пустот в твёрдом скелете используется скалярная величина $\phi(\boldsymbol{x}, t)$ -- \textit{пористость}, определяемая выражением:
\begin{equation}
	\label{porosity}
	\phi(\boldsymbol{x}, t) = \frac{1}{V(\boldsymbol{x}, t)}\int\limits_{V(\boldsymbol{x})}\tilde{\varphi}(\boldsymbol{z}, t) dV,
\end{equation}
	где интеграл берётся по элементарному объёму $V(\boldsymbol{x})$, $\tilde{\varphi}(\boldsymbol{z}, t)$ -- индикаторная функция скелета.

	Наряду с пористостью введём понятия \textit{объёмных долей} флюидов в объёме среды $\phi_F$, $\phi_G$. Для них справедливо соотношение: $\phi_F + \phi_G = \phi_S$, где $\phi_S \equiv \phi$.
	
	\textit{Насыщенностью} пористой среды флюидом $A$ называется величина:
\begin{equation}
	\label{satur}
	S_A = \frac{\phi_A}{\phi}, \quad 0 \leq S_A \leq 1, \quad S_F + S_G = 1.
\end{equation}
	Здесь и далее будем считать: $S \equiv S_F$, $1-S = S_G$.

	Масса пористого насыщенного тела $\beta$ равна:
\begin{align}
	\label{mass}
	m(\beta) &= \int\limits_{\chi(\beta, t)} \rho(\boldsymbol{x}, t) dV = 
	\sum\limits_{A=\{F, G, S\}}\int\limits_{\chi(\beta, t)} \rho_A(\boldsymbol{x}, t) dV,\\
	\rho_A(\boldsymbol{x}, t) &= \phi_A(\boldsymbol{x}, t) \rho_a(\boldsymbol{x}, t), \quad A = \{F, G, S\},
\end{align}
	где $\rho(\boldsymbol{x}, t)$, $\rho_A(\boldsymbol{x}, t)$ -- \textit{осреднённые (эффективные) плотности} насыщенной пористой среды и континуума $A$,
	$\rho_a(\boldsymbol{x}, t)$ -- \textit{истинные плотности} континуума $A$.

	В предположении, что обмен массой между континуумами отсутствует, запишем \textit{локальный закон сохранения массы континуума} $А$ \textit{в форме Лагранжа}:
\begin{equation}
	\label{mass_law_l}
	\rho_{\kappa_A} = \rho_A \left|\det \boldsymbol{F}_A\right|, \quad A=\{F, G, S\},
\end{equation}
	где $\rho_{\kappa_A}$, $\rho_A$ -- плотности массы континуума $A$ в отсчётной и актуальной конфигурациях.
	
	Взяв материальную производную от интегралов в \eqref{mass}, получим \textit{локальное уравнение баланса массы континуума} $A$ \textit{в форме Эйлера}:
\begin{equation}
	\label{mass_law_e}
	\dot{\rho}_A + \rho_A \nabla \cdot \boldsymbol{v}_A = 0, \quad A=\{F, G, S\},
\end{equation}
	или в дивергентной форме:
\begin{equation}
	\label{mass_law_e_div}
	\left.\frac{\partial \rho_A}{\partial t}\right|_{\boldsymbol{x}} + \nabla \cdot \left(\rho_A \boldsymbol{v}_A\right) = 0, \quad A=\{F, G, S\}.
\end{equation}
	
	Выражения \eqref{mass_law_l}, \eqref{mass_law_e}, \eqref{mass_law_e_div} справедливы при отсутствии химических (фазовых) превращений. В противном случае необходимо писать в правой части соответствующие интенсивности переходов $q_A$:
\begin{equation}
	\label{mass_law_e_div1}
	\left.\frac{\partial \rho_A}{\partial t}\right|_{\boldsymbol{x}} + \nabla \cdot \left(\rho_A \boldsymbol{v}_A\right) = q_A, \quad A=\{F, G, S\}.
\end{equation}

\subsection{Напряжения. Законы сохранения импульса и момента импульса.}
	\textit{Силу}, действующую континуум $A$ в объёме тела $\beta$, представим в виде суммы объёмных массовых сил, объёмных сил взаимодействия континуумов и контактных сил:
\begin{equation}
	\label{force_A}
	\boldsymbol{f}_A = \boldsymbol{f}_A^b + \boldsymbol{f}_A^{int} + \boldsymbol{f}_A^c = \int\limits_{\chi({\beta}, t)}\rho_A \boldsymbol{b}_A dV + \int\limits_{\chi({\beta}, t)}\boldsymbol{b}_A^{int} dV + \oint\limits_{\partial\chi({\beta}, t)}\boldsymbol{t}_A dS,
\end{equation}
	где $\boldsymbol{b}_A(\boldsymbol{x}, t)$ -- плотность внешней массовой силы,
	$\boldsymbol{b}_A^{int}$ -- плотность сил, действующих на континуум $A$ со стороны остальных континуумов в элементарном объёме,
	$\boldsymbol{t}_A$ -- контактная сила, действующая на континуум $A$ из вне области $\chi$ со стороны того же континуума.

	Для объёмных сил взаимодействия предполагатся:
\begin{equation}
	\label{force_int}
	\boldsymbol{b}_F^{int} + \boldsymbol{b}_G^{int} + \boldsymbol{b}_S^{int} = 0.
\end{equation}
	
	Сила $\boldsymbol{t}_A$ называется \textit{вектором парциальных напряжений} континуума $A$, задаётся на поверхности и является функцией координат и ориентации поверхности(\textit{постулат Коши}): $\boldsymbol{t}_A = \boldsymbol{t}_A(\boldsymbol{x}, \boldsymbol{n})$. Для вектора $\boldsymbol{t}_A$ справедлива \textit{фундаметальная теорема Коши}:
\begin{equation}
	\label{coushy_th}
	\boldsymbol{t}_A(\boldsymbol{x}, \boldsymbol{n}) = \boldsymbol{\sigma}_A(\boldsymbol{x}) \cdot \boldsymbol{n}, 
\end{equation}
	где тензор $\boldsymbol{\sigma}_A$ называется \textit{тензором эффективных (парциальных) напряжений Коши} для континуума $A$. Для тензора $\boldsymbol{\sigma}_A$ справедливо выражение:
\begin{equation}
	\label{stress_tensor}
	\boldsymbol{\sigma}_A(\boldsymbol{x}, t) = \phi_A(\boldsymbol{x}, t) \boldsymbol{\sigma}_a(\boldsymbol{x}, t),
\end{equation}
	где $\boldsymbol{\sigma}_a(\boldsymbol{x}, t)$ -- \textit{тензор истинных напряжений Коши} для континуума $A$.

	Используя \eqref{coushy_th} и теорему Гаусса-Остроградского запишем \textit{законы сохранения импульса и момента импульса} для континуума $A$ в виде:
\begin{align}
	\label{momentum_law1}
	&\int\limits_{\chi(\beta, t)} \left( \frac{\partial (\rho_A \boldsymbol{v}_A)}{\partial t} 
	+ \nabla \cdot \left( \boldsymbol{v}_A \otimes \rho_A\boldsymbol{v}_A - \boldsymbol{\sigma}^T_A \right) - \rho_A \boldsymbol{b}_A - \boldsymbol{b}_A^{int} \right) dV = 0,\\
	\label{momentum_law2}
	&\int\limits_{\chi(\beta, t)} \left[ \boldsymbol{r} \times \left( \frac{\partial (\rho_A \boldsymbol{v}_A)}{\partial t} 
	+ \nabla \cdot \left( \boldsymbol{v}_A \otimes \rho_A\boldsymbol{v}_A - \boldsymbol{\sigma}^T_A \right) - \rho_A \boldsymbol{b}_A - \boldsymbol{b}_A^{int} \right) + \boldsymbol{\varepsilon} : \boldsymbol{\sigma}_A \right] dV = 0,
\end{align}
	где $\boldsymbol{\varepsilon}$ -- тензор Леви-Чивита. Подставляя \eqref{momentum_law1} в \eqref{momentum_law2} получим:
\begin{equation}
	\label{stress_sym}
	\boldsymbol{\sigma}_A = \boldsymbol{\sigma}_A^T.
\end{equation}
	Для выполнения закона сохранения момента импульса \eqref{momentum_law2} необходимо и достаточно выполнения \eqref{stress_sym}.
	
	Выражения \eqref{momentum_law1}, \eqref{momentum_law2} справедливы, если можно пренебречь вкладом квадрата пульсаций скорости в действующую силу и момент силы. В противном случае в законах сохранения необходимо записывать пульсационные и моментные напряжения.

	Используя \eqref{mass_law_e_div1}, запишем закон сохранения для континуума $A$ в виде:
\begin{equation}
	\label{momentum_law3}
	\rho_A \dot{\boldsymbol{v}}_A + q_A \boldsymbol{v}_A - \nabla \cdot \boldsymbol{\sigma}_A = \rho_A\boldsymbol{b}_A + \boldsymbol{b}_A^{int}.
\end{equation}
	Выражение \eqref{momentum_law3} называется \textit{уравнением движения континуума} $A$.

	Сила взаимодействия, тензор напряжений и объёмная доля представляются в виде:
\begin{equation}
	\label{int_force}
	\boldsymbol{b}_A^{int} = \boldsymbol{b}_A^0 + \boldsymbol{b}_A^{dis}, 
	\quad \boldsymbol{\sigma}_A = \boldsymbol{\sigma}_A^0 + \boldsymbol{\sigma}_A^{dis}, 
	\quad \phi_A = \phi_A^0 + \phi_A^{dis},	
	\quad A = \{F, G, S\},
\end{equation}
	где $\boldsymbol{b}_A^0$, $\boldsymbol{\sigma}_A^0$, $\phi_A^0$ и $\boldsymbol{b}_A^{dis}$, $\boldsymbol{\sigma}_A^{dis}$, $\phi_A^{dis}$ -- \textit{равновесные} и \textit{диссипативные} значения силы взаимодействия, тензора напряжений и объёмной доли. При этом имеем: $\boldsymbol{b}_A^0 = \boldsymbol{\sigma}_a \cdot \nabla (\phi_A)$. Для $\boldsymbol{b}_A^{dis}$ формулируются определяющие соотношения, которые приводят к закону Дарси.
	
	Cуммируя \eqref{momentum_law3} по всем континуумам $A = \{F, G, S\}$, получим:
\begin{align}
	\label{momentum_law_full}
	\rho_A \dot{\boldsymbol{v}} + \nabla \cdot \left(\sum\limits_{A}\left(\boldsymbol{w}_A \otimes \rho_A\boldsymbol{w}_A\right) -\boldsymbol{\sigma}\right) = \rho \boldsymbol{b},\\
	\rho = \sum\limits_{A} \rho_A, \quad \rho\boldsymbol{v}=\sum\limits_{A}\rho_A\boldsymbol{v}_A, \quad 
	\rho\boldsymbol{b}=\sum\limits_{A}\rho_A\boldsymbol{b}_A\\
	\boldsymbol{w}_A = \boldsymbol{v} - \boldsymbol{v}_A, \quad \sum\limits_A \rho_A\boldsymbol{w}_A = 0, \quad
	\boldsymbol{\sigma} = \sum\limits_{A} \boldsymbol{\sigma}_A,
\end{align}
	где $\rho$ -- плотность среды, $\rho\boldsymbol{v}$ -- среднемассовая (барицентрическая) скорость,
	$\boldsymbol{w}_A$ -- относительные (диффузионные) скорости континуума $A$,
	$\boldsymbol{\sigma}$ -- тензор полных напряжений среды.
	
\subsection{Закон сохранения энергии. Второе начало термодинамики.}

	Для описание теплопереноса в рассматриваемой системе воспользуемся гипотезой \textit{о локальном термодинамическом равновесии}, которая предполагает, что внутри каждого элементарного объёма среды все континуумы находятся в состоянии термодинамического равновесия. Считается, что при малых скоростях движения флюидов и высокой теплопроводности, использование гипотезы оправдано. В таком случае можно ввести понятие \textit{температуры} $\theta(\boldsymbol{x}, t)$ для материальной точки $\boldsymbol{x}$ в момент времени $t$.

	Запишем скорость подвода тепла к континууму $A$ в объёме тела $\beta$:
\begin{equation}
	\label{heat_supply}
	Q_A = \int\limits_{\chi(\beta, t)}(\rho_A r_A + r_A^{int})dV + \oint\limits_{\partial \chi(\beta, t)} h_A dS,
\end{equation}
	гле $r_A$ -- \textit{плотность внешних} (по отношению к континууму $A$) \textit{объёмных источников тепла},
	$h_A$ -- \textit{поверхностный приток тепла},
	$r_A^{int}$ -- \textit{скорость объёмного теплообмена}.

	Величины $r_A^{int}$, $h_A$ во многом аналогичны характеристикам напряженного состояния $\boldsymbol{b}_A^{int}$, $\boldsymbol{t}_A$. Для $r_A^{int}$ справедливо соотношение:
\begin{equation}
	\label{heat_exchange}
	r_F^{int} + r_G^{int} + r_S^{int} = 0,
\end{equation}
	а для $h_A$ справедлива \textit{фундаментальная теорема Фурье-Стокса}:
\begin{equation}
	\label{fourier_stocks}
	h_A(\boldsymbol{x}, \boldsymbol{n}) = \boldsymbol{q}_A(\boldsymbol{x}) \cdot \boldsymbol{n},
\end{equation}
	где $\boldsymbol{q}_A(\boldsymbol{x})$ -- \textit{вектор теплового потока}.

	Запишем закон баланса энергии в виде:
\begin{align}
	\label{energy_law1}
	\frac{d}{dt}\int\limits_{\chi(\beta, t)} \rho e dV &= \sum_A\left[\int\limits_{\chi(\beta, t)}\left(\rho_A\boldsymbol{b}_A\cdot\boldsymbol{v}_A\right)dV + \oint\limits_{\partial\chi(\beta, t)}\left(\boldsymbol{t}_A \cdot \boldsymbol{v}_A\right)dS\right] + \int\limits_{\chi(\beta, t)}\left(\nabla \cdot \boldsymbol{q} + \rho r\right)dV\\
	\label{energy_law2}
	\rho e &= \sum_A \rho_A e_A, \quad e_A = u_A + \frac{\boldsymbol{v}_A \cdot \boldsymbol{v}_A}{2}, \quad \boldsymbol{q} = \sum\limits_A \boldsymbol{q}_A,\\
	\label{energy_law3}
	\rho r &= \sum\limits_{A} \rho_A r_A, \quad A = \{F, G, S\},
\end{align}
	где $e_A$ -- \textit{плотность полной энергии} системы,
	$u_A$ -- \textit{плотность внутренней энергии}.
	В выражении \eqref{energy_law1} слева стоит материальная производная от полной энергии тела $\beta$ -- внутренней и кинетической, справа стоит мощность внешних сил и суммарная скорость подвода тепла.

	Используя \eqref{mass_law_e_div1}, \eqref{momentum_law3} получим \textit{приведённое уравнение баланса энергии} в виде:
\begin{equation}
	\label{energy_law4}
	\sum\limits_{A}\left[\rho_A \dot{u}_A + q_A\left(u_A - \frac{\boldsymbol{v}_A \cdot \boldsymbol{v}_A}{2}\right)\right] = 
	\sum\limits_{A}\left[\boldsymbol{b}_A^{int}\cdot \boldsymbol{w}_A + \boldsymbol{\sigma}_A : (\nabla \otimes \boldsymbol{v}_A)\right] + 
	\nabla \cdot \boldsymbol{q} + \rho r,
\end{equation}
	где $\boldsymbol{w}_A$ -- скорость материальной точки континуума $A$ относительно скелета $S$.

	Для формулировки второго начала термодинамики введём понятие плотности \textit{энтропии} континуума $A$ -- $\eta_A$.
	Введение понятия энтропии позволяет верным образом учесть направление термодиначеских процессов, а также ввести понятие ноебратимых процессов.
	Между тем второе начало термодинамики будет использовано далее для построения определяющих соотношений насыщенной пористой среды.

	Запишем \textit{второе начало термодинамики} в интегральном виде:
\begin{equation}
	\label{second_origin1}
	\frac{d}{dt}\int\limits_{\chi(\beta,t)}\sum\limits_A \rho_A \eta_A dV \geq \int\limits_{\chi(\beta, t)}\frac{\rho r}{\theta}dV + \oint\limits_{\partial\chi(\beta, t)}\frac{\boldsymbol{q}\cdot \boldsymbol{n}}{\theta}dS.
\end{equation}

	Преобразуем \eqref{second_origin1} и запишем в дифференциальном виде:
\begin{align}
	\label{second_origin2}
	\theta\sum\limits_{A}\left[\rho_A \dot{\eta}_A + q_A \eta_A\right] &- \nabla \cdot \boldsymbol{q} - \rho r + \frac{\boldsymbol{q}\cdot\nabla\theta}{\theta} \geq 0,\\
	\delta_M = \theta\sum\limits_{A}\left[\rho_A \dot{\eta}_A + q_A \eta_A\right] &- \nabla \cdot \boldsymbol{q} - \rho r, \quad \delta_T = \frac{\boldsymbol{q}\cdot\nabla\theta}{\theta},\\
	\delta_M &+ \delta_T \geq 0,
\end{align}
	где $\delta_M$ имеет смысл \textit{механической (внутренней) диссипации}, 
	а $\delta_T$ -- \textit{термической диссипации}, связанной с теплопроводностью среды.

	Запишем \eqref{second_origin2} через \textit{плотность свобоной энергии} $\psi_A = u_A - \theta \eta_A$. Для этого воспользуемся \eqref{velocities_strains}, \eqref{energy_law4}, \eqref{int_force}, получим:
\begin{multline}
	\label{second_origin3}
	-\sum\limits_{A} \left[\rho_A \left(\dot{\psi_A} + \eta_A\dot{\theta}\right) + q_A\left(\psi_A - \frac{\boldsymbol{v}_A\cdot \boldsymbol{v}_A}{2}\right) + \boldsymbol{\sigma}_a :\left(\nabla \otimes \phi_A^0 \boldsymbol{w}_A\right)\right]+\\
	+\left(\boldsymbol{\sigma} \cdot \boldsymbol{F}_S^{-1T}\right) : \dot{\boldsymbol{F}}_S + \delta_f + \delta_T \geq 0
\end{multline}
\begin{equation}
	\label{second_origin4}
	\boldsymbol{\sigma} = \sum_A \boldsymbol{\sigma}_A, 
	\quad \delta_f = \sum_A\left[\boldsymbol{\sigma}_A^{dis}:\left(\nabla \otimes \boldsymbol{w}_A\right)+\boldsymbol{b}_A^{dis} \cdot \boldsymbol{w}_A\right],
\end{equation}
	где $\boldsymbol{\sigma}$ -- тензор полных напряжений среды,
	$\delta_f$ -- \textit{диссипация фильтрации}.
	Выражение \eqref{second_origin4} носит название -- \textit{вторая форма приведённого неравенства Клаузиуса-Дюгема}.
	
\subsection{Определяющие уравнения.}

	 