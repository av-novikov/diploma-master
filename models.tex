\section*{Глава 1\\Основы механики и термодинамики насыщенных пористых сред}
\addcontentsline{toc}{section}{Глава 1. Основы механики и термодинамики насыщенных пористых сред}
\setcounter{section}{1}
\setcounter{subsection}{0}
\setcounter{equation}{0}

	Насыщенная пористая среда -- совокупность твёрдого деформируемого скелета и флюида, насыщающего этот скелет.
	Под \textit{флюидом} понимается смесь жидкостей и газов, способная перемещаться внутри порового пространства скелета.
	Для описания совместного движения скелета и флюида используется \textit{гипотеза суперпозиции континуумов}, которая предполагает что в каждой точке пространства находится и скелет, и флюид. 
	
	В данной главе будут представлены законы сохранения, основные поределяющие соотношения такой системы. Как результат будут получены математические модели 	процессов массо- и теплопереноса в пористых средах.
	
	Подробное изложение законов сохранения, постоение определяющих соотношений для таких системы можно найти в \cite{basniev, fortov, kondaurov, checkalyuk}.

\subsection{Кофигурации. Градиент деформации. Подходы Эйлера и Лагранжа описания движения сплошной среды.}
	\textit{Материальная точка} или \textit{элементарный объём} -- объём сплошной среды, пренебрежимо малый по сравнению с размерами рассматриваемой задачи, но, при том,   достаточный для того чтобы можно было проводить по нему осреднение. Дальнейшее рассмотрение будет продится именно для таких объёмов.
	