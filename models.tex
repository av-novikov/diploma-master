\section*{Глава 1\\Основы механики и термодинамики насыщенных пористых сред}
\addcontentsline{toc}{section}{Глава 1. Основы механики и термодинамики насыщенных пористых сред}
\setcounter{section}{1}
\setcounter{subsection}{0}
\setcounter{equation}{0}

	Насыщенная пористая среда -- совокупность твёрдого деформируемого скелета и флюида, насыщающего этот скелет.
	Под \textit{флюидом} понимается смесь жидкостей и газов, способная перемещаться внутри порового пространства скелета.
	Для описания совместного движения скелета и флюида используется \textit{гипотеза суперпозиции континуумов}, которая предполагает что в каждой точке пространства находится и скелет, и флюид. 

	Флюид, в свою очередь, может быть многофазным и многокомпонентным. Здесь и далее будем отождествлять понятие \textit{компоненты} с химическим веществом, входящим в состав флюида. \textit{Фазой} будем называть термодинамически равновесное состояние вещества, качественно отличное от других равновесных состояний того же вещества. Подробное изложение законов сохранения, построение определяющих соотношений для таких систем можно найти в \cite{basniev, fortov, kondaurov, checkalyuk}.
	Здесь же, для простоты, будем рассматривать флюид состоящий из одной компоненты, которая может находится в двух фазах -- жидкой и газообразной.
	
	В данной главе будут представлены законы сохранения, основные определяющие соотношения такой системы. Как результат будут получены математические модели 	процессов массо- и теплопереноса в пористых средах.

\subsection{Кофигурации. Градиент деформации. Подходы Эйлера и Лагранжа описания движения сплошной среды.}
	\textit{Материальная точка} или \textit{элементарный объём} -- объём сплошной среды, пренебрежимо малый по сравнению с размерами рассматриваемой задачи, но, при том,   достаточный для того чтобы можно было проводить по нему осреднение. Дальнейшее рассмотрение будет проводится именно для таких объёмов.
	
	Обозначим $\kappa_A \in \mathbb{E}^3$ -- область, которую занимают частицы скелета (A=S) или флюида (A=F, G) в момент времени $t=0$. Область $\kappa_A$ в дальнейшем будем называть \textit{отсчётной (начальной) конфигурацией} скелета или флюида соответственно. 
	Область $\chi(t) \in \mathbb{E}^3$, занятую в момент времени $t > 0$ частицами скелета и флюида, назовём \textit{актуальной} или \textit{текущей конфигурацией}. Отображения $\kappa_A \to \chi(t)$ будем называть деформацией концинуума $A$.
	
	Здесь и далее предполагается, что области $\kappa_A$, $\chi(t)$ -- регулярны, отображения $\kappa_A \to \chi(t)$ -- кусочно-гомеоморфны и дифференцируемы. Тогда существуют взаимнооднозначные дифференцируемые связи:
\begin{equation}
	\label{motion_law}
	\mathbf{x} = \mathbf{x}(\mathbf{X}_A, t), \quad t > 0,\quad \mathbf{x} \in \chi(t), \quad \mathbf{X}_A \in \kappa_A,
\end{equation}
	которые называются \textit{законами движения} материальных точек скелета и флюида.
	
	Возмём дифференциал от \eqref{motion_law}:
\begin{align}
	\label{gradient}
	d\mathbf{x} = d\mathbf{X}_A \cdot \left(\nabla_{\kappa} \otimes \mathbf{x} \right) &= d\mathbf{X} \cdot \mathbf{F}_A^T = \mathbf{F}_A \cdot d\mathbf{X},\nonumber\\
	\mathbf{F}_A(\mathbf{X}, t) &= \left[\nabla_{\kappa} \otimes \mathbf{x}(\mathbf{X}_A, t)\right]^T,
\end{align}
	где $\mathbf{F}_A$ -- тензор второго ранга, называемый \textit{градиентом деформации (дисторсией)} континуума A.
	
	Для градиента деформаций $\textbf{F}_A$ справедлива \textit{теорема Коши о полярном разложении}, которая позволяет представить деформацию элемента $d\mathbf{X}_A$ как комбинацию растяжения (сжатия) и вращения как жесткого целого:
\begin{equation}
	\label{coushy_th}
	\mathbf{F}_A = \mathbf{R}_A \cdot \mathbf{U}_A = \mathbf{V}_A \cdot \mathbf{R}_A,
\end{equation}
	где $\mathbf{R}_A$ -- ортогональный тензор второго ранга, называемый \textit{тензором поворота},
	$\mathbf{U}_A, \mathbf{V}_A$ -- симметричные положительной определённые тензоры второго ранга, называемые \textit{правым и левым тензорами растяжения}. Разложение \eqref{coushy_th} единственно.
	
	Частной производной закона движения \eqref{motion_law} по времени является \textit{вектор скорости материальной точки}:
\begin{equation}
	\label{velocity}
	\mathbf{v}_A(\mathbf{X}_A, t) \equiv \dot{\mathbf{x}}(\mathbf{X}_A, t) = \left.\frac{\partial\mathbf{x}(\mathbf{X}, t)}{\partial t}\right|_{\mathbf{X}_A}.
\end{equation}
	Здесь и далее точкой будем обозначать \textit{материальную производную по времени} (при постоянном $\mathbf{X}_A$).
	
	Описание характристик материальной точки функциями от $\mathbf{X}_A$ носит название \textit{материального} или \textit{лагранжевого описания среды}, а радиус-вектор $\textbf{X}_A$ носит название \textit{материальной} или \textit{лагранжевой переменной}. Если же характеристики представляются функциями $\mathbf{x}$, то такой подход называется \textit{пространственным} или \textit{эйлеровым описанием среды}, переменная $\mathbf{x}$ -- \textit{пространственной} или \textit{эйлеровой переменной}.

\subsection{Тезоры деформаций. Уравнение совместности скоростей и деформаций.}
	Для того чтобы охарактеризовать деформации континуума вводятся специальные меры -- \textit{тензоры конечных деформаций}. Наиболее употребительными являются \textit{тензоры Коши-Грина} $\mathbf{E}_A$ \textit{и Альманзи} $\mathbf{A}_A$:
\begin{align}
	\label{strain_measures}
	\mathbf{E}_A &= \frac{1}{2}\left(\mathbf{F}_A^T \cdot \mathbf{F}_A - \mathbf{I}\right)\\
	\mathbf{A}_A &= \frac{1}{2}\left(\mathbf{I} - \mathbf{F}_A^{-1T} \cdot \mathbf{F}_A^{-1}\right).
\end{align}

	Представляя градиент деформаций \eqref{gradient} через вектор перемещений $\mathbf{u}_A = \mathbf{x}_A - \mathbf{X}_A$, подставляя в \eqref{strain_measures} и пренебрегая членами второго порядка малости, получим \textit{тензор малых деформаций} $\mathbf{e_A}$:
\begin{equation}
	\label{small_strains}
	\mathbf{e}_A = \frac{1}{2}\left(\left(\nabla \otimes \mathbf{u}\right) + \left(\nabla\otimes \mathbf{u}\right)^T\right),
\end{equation}
	где в принятых допущениях: $\nabla_{\kappa} \simeq \nabla$.

	Величины $\mathbf{F}_A(\mathbf{X}_A, t)$ и $\mathbf{v}_A(\mathbf{X}_A, t)$ являются первыми производными отображения $\kappa_A \to \chi(t)$. Предполагая отображение \eqref{motion_law} кусочно дважды непрерывно-дифференцируемым, получим соотношение:
\begin{equation}
	\label{velocities_strains}
	\dot{\mathbf{F}}_A = \left(\nabla_{\kappa} \otimes \mathbf{v}_A \right)^T,
\end{equation}
	называемое \textit{уравнением совместности скоростей и деформаций}.

\subsection{Пористость. Эффективная и истинная плотности. Закон сохранения масс.}

	Для описания доли пустот в твёрдом скелете используется скалярная величина $\phi(\mathbf{x}, t)$ -- \textit{пористость}, определяемая выражением:
\begin{equation}
	\label{porosity}
	\phi(\mathbf{x}, t) = \frac{1}{V(\mathbf{x}, t)}\int\limits_{V(\mathbf{x})}\tilde{\varphi}(\mathbf{z}, t) dV,
\end{equation}
	где интеграл берётся по элементарному объёму $V(\mathbf{x})$, $\tilde{\varphi}(\mathbf{z}, t)$ -- индикаторная функция скелета.

	Наряду с пористостью введём понятия объёмных долей флюидов в объёме среды $\phi_F$, $\phi_G$. Для них справедливо соотношение: $\phi_F + \phi_G = \phi$.
	
	\textit{Насыщенностью} пористой среды флюидом $A$ называется величина:
\begin{equation}
	\label{satur}
	S_A = \frac{\phi_A}{\phi}, \quad 0 \leq S_A \leq 1, \quad S_F + S_G = 1.
\end{equation}
	Здесь и далее будем считать: $S \equiv S_F$, $1-S = S_G$.

	Масса пористого насыщенного тела $\beta$ равна:
\begin{align}
	\label{mass}
	m(\beta) = \int\limits_{\chi(\beta, t)} \rho(\mathbf{x}, t) dV &= 
	\int\limits_{\chi(\beta, t)} \rho_F(\mathbf{x}, t) dV + \int\limits_{\chi(\beta, t)}\rho_G(\mathbf{x}, t)dV + \int\limits_{\chi(\beta, t)}\rho_S(\mathbf{x}, t)dV,\\
	\rho_F(\mathbf{x}, t) &= \phi(\mathbf{x}, t) S(\mathbf{x}, t) \rho_f(\mathbf{x}, t),\\
	\rho_G(\mathbf{x}, t) &= \phi(\mathbf{x}, t)(1-S(\mathbf{x}, t))\rho_g(\mathbf{x}, t),\\
	\rho_S(\mathbf{x}, t) &= (1-\phi(\mathbf{x}, t))\rho_s(\mathbf{x}, t),
\end{align}
	где $\rho(\mathbf{x}, t)$, $\rho_F(\mathbf{x}, t)$, $\rho_G(\mathbf{x}, t)$, $\rho_S(\mathbf{x}, t)$ -- \textit{осреднённые (эффективные) плотности} насыщенной пористой среды, фаз флюида и скелета соответственно,
	$\rho_f(\mathbf{x}, t)$, $\rho_g(\mathbf{x}, t)$, $\rho_s(\mathbf{x}, t)$ -- \textit{истинные плотности} фаз флюида и скелета.

	В предположении, что обмен массой между континуумами отсутствует, запишем \textit{локальный закон сохранения массы континуума} $А$ \textit{в форме Лагранжа}:
\begin{equation}
	\label{mass_law_l}
	\rho_{\kappa_A} = \rho_A \left|\det \mathbf{F}_A\right|, \quad A=\{F, G, S\},
\end{equation}
	где $\rho_{\kappa_A}$, $\rho_A$ -- плотности массы континуума $A$ в отсчётной и актуальной конфигурациях.
	
	Взяв материальную производную от интегралов в \eqref{mass}, получим \textit{локальное уравнение баланса массы континуума} $A$ \textit{в форме Эйлера}:
\begin{equation}
	\label{mass_law_e}
	\dot{\rho}_A + \rho_A \nabla \cdot \mathbf{v}_A = 0, \quad A=\{F, G, S\},
\end{equation}
	или в дивергентной форме:
\begin{equation}
	\label{mass_law_e_div}
	\left.\frac{\partial \rho_A}{\partial t}\right|_{\mathbf{x}} + \nabla \cdot \left(\rho_A \mathbf{v}_A\right) = 0, \quad A=\{F, G, S\}.
\end{equation}
	
	Выражения \eqref{mass_law_l}, \eqref{mass_law_e}, \eqref{mass_law_e_div} справедливы при отсутствии химических (фазовых) превращений. В противном случае необходимо писать в правой части соответствующие интенсивности переходов:
\begin{equation}
	\label{mass_law_e_div}
	\left.\frac{\partial \rho_A}{\partial t}\right|_{\mathbf{x}} + \nabla \cdot \left(\rho_A \mathbf{v}_A\right) = q_A, \quad A=\{F, G, S\}.
\end{equation}

\subsection{Напряжения. Законы сохранения импульса и момента импульса.}
	\textit{Силу}, действующую континуум $A$ в объёме тела $\beta$, представим в виде суммы объёмных массовых сил, объёмных сил взаимодействия континуумов и контактных сил:
\begin{equation}
	\label{force_A}
	\mathbf{f}_A = \mathbf{f}_A^b + \mathbf{f}_A^{int} + \mathbf{f}_A^c = \int\limits_{\chi({\beta}, t)}\rho_A \mathbf{g} dV + \int\limits_{\chi({\beta}, t)}\mathbf{b}_A^{int} dV + \oint\limits_{\partial\chi({\beta}, t)}\mathbf{t}_A dS,
\end{equation}
	где $\mathbf{g}(\mathbf{x}, t)$ -- плотность внешней массовой силы,
	$\mathbf{b}_A^{int}$ -- плотность сил, действующих на континуум $A$ со стороны остальных континуумов в элементарном объёме,
	$\mathbf{t}_A$ -- контактная сила, действующая на континуум $A$ из вне области $\chi$ со стороны того же континуума.

	Для объёмных сил взаимодействия предполагатся:
\begin{equation}
	\label{force_int}
	\mathbf{b}_F^{int} + \mathbf{b}_G^{int} + \mathbf{b}_S^{int} = 0.
\end{equation}
	
	Сила $\mathbf{t}_A$ называется \textit{вектором парциальных напряжений} континуума $A$, задаётся на поверхности и является функцией координат и ориентации поверхности(\textit{постулат Коши}): $\mathbf{t}_A = \mathbf{t}_A(\mathbf{x}, \mathbf{n})$. Для вектора $\mathbf{t}_A$ справедлива \textit{фундаметальная теорема Коши}:
\begin{equation}
	\label{coushy_th}
	\mathbf{t}_A(\mathbf{x}, \mathbf{n}) = \mathbf{\sigma}_A(\mathbf{x}) \cdot \mathbf{n}, 
\end{equation}
	где тензор $\mathbf{\sigma}_A$ называется \textit{тензором парциальных напряжений Коши} для континуума $A$.

	Учитывая всё выше сказанное, а также \eqref{mass_law_e}, запишем \textit{закон сохранения импульса} для континуума $A$ в виде:
\begin{equation}
	\label{momentum_law}
	\rho_A\frac{d_A\mathbf{v}_A}{dt} - \nabla \cdot \sigma_A = \rho_A\mathbf{g} + \mathbf{b}_A^{int}
\end{equation}

