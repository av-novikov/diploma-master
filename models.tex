\section*{Глава 1\\Основы механики и термодинамики насыщенных пористых сред}
\addcontentsline{toc}{section}{Глава 1. Основы механики и термодинамики насыщенных пористых сред}
\setcounter{section}{1}
\setcounter{subsection}{0}
\setcounter{equation}{0}

	Насыщенная пористая среда -- совокупность твёрдого деформируемого скелета и флюида, насыщающего этот скелет.
	Под \textit{флюидом} понимается смесь жидкостей и газов, способная перемещаться внутри порового пространства скелета.
	Для описания совместного движения скелета и флюида используется \textit{гипотеза суперпозиции континуумов}, которая предполагает что в каждой точке пространства находится и скелет, и флюид. 
	
	В данной главе будут представлены законы сохранения, основные определяющие соотношения такой системы. Как результат будут получены математические модели 	процессов массо- и теплопереноса в пористых средах.
	
	Подробное изложение законов сохранения, постоение определяющих соотношений для таких системы можно найти в \cite{basniev, fortov, kondaurov, checkalyuk}.

\subsection{Кофигурации. Градиент деформации. Подходы Эйлера и Лагранжа описания движения сплошной среды.}
	\textit{Материальная точка} или \textit{элементарный объём} -- объём сплошной среды, пренебрежимо малый по сравнению с размерами рассматриваемой задачи, но, при том,   достаточный для того чтобы можно было проводить по нему осреднение. Дальнейшее рассмотрение будет проводится именно для таких объёмов.
	
	Обозначим $\kappa_A \in \mathbb{E}^3$ -- область, которую занимают частицы скелета (A=S) или флюида (A=F) в момент времени $t=0$. Область $\kappa_A$ в дальнейшем будем называть \textit{отсчётной (начальной) конфигурацией} скелета или флюида соответственно. 
	Область $\chi(t) \in \mathbb{E}^3$, занятую в момент времени $t > 0$ частицами скелета и флюида, назовём \textit{актуальной} или \textit{текущей конфигурацией}. Отображения $\kappa_A \to \chi(t)$ будем называть деформацией концинуума $A$.
	
	Здесь и далее предполагается, что области $\kappa_A$, $\chi(t)$ -- регулярны, отображения $\kappa_A \to \chi(t)$ -- кусочно-гомеоморфны и дифференцируемы. Тогда существуют взаимнооднозначные дифференцируемые связи:
\begin{equation}
	\label{motion_law}
	\mathbf{x} = \mathbf{x}(\mathbf{X}_A, t), \quad t > 0,\quad \mathbf{x} \in \chi(t), \quad \mathbf{X}_A \in \kappa_A,
\end{equation}
	которые называются \textit{законами движения} материальных точек скелета и флюида.
	
	Возмём дифференциал от \eqref{motion_law}:
\begin{equation}
	\label{gradient}
	d\mathbf{x} = d\mathbf{X}_A \cdot \left(\nabla_{\kappa} \otimes \mathbf{x} \right) = d\mathbf{X} \cdot \mathbf{F}_A^T = \mathbf{F}_A \cdot d\mathbf{X},
\end{equation}
	где $\mathbf{F}_A$ -- тензор второго ранга, называемый \textit{градиентом деформации (дисторсией)} континуума A.
	
	Для градиента деформаций $\textbf{F}_A$ справедлива \textit{теорема Коши о полярном разложении}, которая позволяет представить деформацию элемента $d\mathbf{X}_A$ как комбинацию растяжения (сжатия) и вращения как жесткого целого:
\begin{equation}
	\label{coushy_th}
	\mathbf{F}_A = \mathbf{R}_A \cdot \mathbf{U}_A = \mathbf{V}_A \cdot \mathbf{R}_A,
\end{equation}
	где $\mathbf{R}_A$ -- ортогональный тензор второго ранга, называемый \textit{тензором поворота},
	$\mathbf{U}_A, \mathbf{V}_A$ -- симметричные положительной определённые тензоры второго ранга, называемые \textit{правым и левым тензорами растяжения}. Разложение \eqref{coushy_th} единственно.
	
	Описание характристик материальной точки функциями от $\mathbf{X}_A$ носит название \textit{материального} или \textit{лагранжевого описания среды}, а радиус-вектор $\textbf{X}_A$ носит название \textit{материальной} или \textit{лагранжевой переменной}. Если же характеристики представляются функциями $\mathbf{x}$, то такой подход называется \textit{пространственным} или \textit{эйлеровым описанием среды}, переменная $\mathbf{x}$ -- \textit{пространственной} или \textit{эйлеровой переменной}.

\subsection{Тезоры деформаций. Уравнение совместности скоростей и деформаций.}
	Для того чтобы охарактеризовать деформации континуума вводятся специальные меры -- \textit{тензоры конечных деформаций}. Наиболее употребительными являются \textit{тензоры Коши-Грина} $\mathbf{E}_A$ \textit{и Альманзи} $\mathbf{A}_A$:
\begin{align}
	\label{strain_measures}
	\mathbf{E}_A &= \frac{1}{2}\left(\mathbf{F}_A^T \cdot \mathbf{F}_A - \mathbf{I}\right)\\
	\mathbf{A}_A &= \frac{1}{2}\left(\mathbf{I} - \mathbf{F}_A^{-1T} \cdot \mathbf{F}_A^{-1}\right)
\end{align}

\subsection{Закон сохранения масс.}