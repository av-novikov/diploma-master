\section*{Введение}
\addcontentsline{toc}{section}{Введение}
\setcounter{subsection}{0}
	
	Традиционно расчёт фильтрации флюидов в нефтегазоносных пластах проводят пренебрегая изменением температуры среды. Во-первых, потому, что изменения температуры при фильтрации крайне малы, и не оказывают сколько-нибудь заметного влияния на свойства среды. Во-вторых, для записи температурных данных необходимы высокоточные датчики, которые получили широкое распространение лишь в последнее десятилетие. В-третьих, при дополнительном расчёте теплопереноса возникает необходимость указания большого количества характеристик (как компонентов среды, так и задачи в целом), значения которых неизвестны, либо известны с весьма плохой точностью. В этой связи, модели неизотермической фильтрации при гидродинамическом моделировании месторождений остаются не у дел.

	Тем не менее существует ряд задач, в которых привлечение термодинамики насыщенной пористой среды позволяет определить некоторые эксплуатационные характеристики пласта. Речь идет о измерении \textit{давления}, \textit{дебита} и \textit{температуры} в стволе скважины и последующей совместной интерпретации этих данных на основе соответствующих моделей массо- и теплопереноса.
Такие исследования носят название \textit{термогидродинамических}, и подразделяются на два вида:
\begin{itemize}
	\item измерение параметров вдоль ствола скважины;
	\item измерение параметров во времени на определённой глубине.
\end{itemize}