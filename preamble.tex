\section*{Введение}
\addcontentsline{toc}{section}{Введение}
\setcounter{subsection}{0}
	
	Традиционно расчёт фильтрации флюидов в нефтегазоносных пластах проводят пренебрегая изменением температуры среды. Во-первых, потому, что изменения температуры при фильтрации крайне малы, и не оказывают сколько-нибудь заметного влияния на свойства среды. Во-вторых, для записи температурных данных необходимы высокоточные датчики, которые получили широкое распространение лишь в последнее десятилетие. В-третьих, при дополнительном расчёте теплопереноса возникает необходимость указания большого количества характеристик (как компонентов среды, так и задачи в целом), значения которых неизвестны, либо известны с весьма плохой точностью. В этой связи, модели неизотермической фильтрации при гидродинамическом моделировании месторождений остаются не у дел.

	Тем не менее существует ряд задач, в которых привлечение термодинамики насыщенной пористой среды позволяет определить некоторые \textit{эксплуатационные характеристики пласта}. Речь идет о измерении \textit{давления}, \textit{дебита} и \textit{температуры} в стволе скважины и последующей совместной интерпретации этих данных на основе соответствующих моделей массо- и теплопереноса.
Такие исследования носят название \textit{термогидродинамических}, и подразделяются на два вида:
\begin{itemize}
	\item измерение параметров вдоль ствола скважины;
	\item измерение параметров во времени на определённой глубине.
\end{itemize}
	
	Первый тип исследования относится к стандартным \textit{геофизическим} исследованиям, проводимым на этапе освоения и эксплуатации скважин. Их результатом является определение следующих хакрактеристик пласта:
\begin{itemize}
	\item выявление работающих пластов;
	\item определение интервалов притока;
	\item оценка состава флюида, определение интервалов притока воды, нефти и газа;
	\item определение дебита, обводнённости, коэффициента продуктивности;
	\item определение пластового давления и температуры.
\end{itemize}
	
	Результаты исследований по второму типу представляют особую ценность, т.к. позволяют определить динамические характеристики экплуатируемых пластов. 
	Широко распространённой техникой являются, так называемые, \textit{гидродинамические исследования} (ГДИ). 
	Этот метод позволяет, на основе интерпретации кривых востановления уровня (КВУ), кривых востановления давления (КВД), определить проницаемость пласта, суммарный  скин-фактор, продуктивность скважины, которая, в свою очередь, является основным показателем работы скважины.
	Тем не менее, метод не даёт никаких более подробных сведений о структуре околоскважинной зоны пласта (ОЗП), тогда как с гидродинамической точки зрения эта область представляет наибольший интерес.

	Начиная с первичного вскрытия пласта, когда происходит проникновение фильтрата бурового раствора, формируется сложная структура ОЗП, в результате чего, меняются и её фильтрационно-ёмкостные свойства (ФЕС). На этапе заканчивания, пласт прошивается системой перфорационных каналов, которая существенно меняет картину теченения в ОЗП. Эти и другие технологические процедуры серьёзным образом усложняют моделирование многофазного притока к скважине.
	
	Для определения структуры ОЗП, в последнее время начинают применяться \textit{термогидродинамические исследования} (ТГДИ) скважин \cite{ramazanov_diss,ramazanov_spe, ramazanov_spe1, posv}, основа которых была заложена ещё более полувека тому назад Э.Б. Чекалюком \cite{checkalyuk}.
	Эти исследования основаны на моделях неизотермической фильтрации, с помощью которых расчитывается забойная температура.
	Они используются для интерпретации промысловых данных, дополняя развитую технику интерпретации ГДИ.

	Поскольку рассматриваемые исследования основаны на интерпретации нестационарных данных, принципиальным моментом здесь является скорость процесса.
	Давление очень быстро реагирует на изменение режима добычи и не позволяет получить информацию о ОЗП.
	В свою очередь, скорость основных тепловых процессов существенно ниже, что даёт возможность использовать ТГДИ для исследования ОЗП.

	На пути прогресса распространения этой методики долгое время стояла недостаточная разрешающая способность темпертурных датчиков.
	Современные же устройства позволяют определять температуру с точностью до 0,0001 K, что является достаточным для идентификации даже незначительных изменений температурного фона в стволе скважины.

	Тем не менее стоит отметить, что непосредственное применение ТГДИ на практике возможно лишь при наличии соотвествующего оборудования и его правильной установки, проведении исследований тепловых PVT свойств флюида и породы, определённых системах заканчивания скважины, вкупе с другими геофизическими и промысловыми исследованиями.

	В представленной работе, посвящённой математическому моделированию процессов, лежащих в основе ТГДИ, сделан упор на рассмотрение фазовых переходов происходящих при работе нефтяной скважины ниже давления насыщения, учёта системы вторичного вскрытия (перфорацинных каналов) при исследовании динамики забойной температуры.
	
	В Главе 1 представлены основные соотношения механики и термодинамики насыщенных пористых сред. Произведён последовательный вывод уравнений для рассматриваемой модели.
	В Главе 2 ...

\subsection*{Обзор литературы}
	Подробный вывод и описание законов сохранения и определяющих соотношений механики и термодинамики насыщенной пористой среды можно найти в \cite{kondaurov}.
	Описание процессов тепло- и массо- переноса в многофазных многокомпонентных средах представлено в \cite{nigmatulin, multiphase}. 
	Основные задачи фильтрации флюидов в нефтегазоносных пластах рассмотрены в \cite{basniev, charniy}.
	Модели неизотермической фильтрации представлены в \cite{checkalyuk, alishaev}.
	Уравнения состояния и расчёт констант фазового равновесия и PVT хар-к многофазной многокомпонентной углеводородной смеси рассмотрены в \cite{brusilovskiy}.

	Математической моделью задач тепло- и массопереноса в пористых средах, как правило, является смешанная задача для системы нелинейных уравнений в частных производных. Для однородной прямоугольной области области задача массопереноса имеет аналитическое решение через функцию Грина \cite{vladimirov}.
	Применение функций Грина для расчёта притока к скважинам произвольной геометрии рассмотрено в \cite{aziz_green}.
	В \cite{posv1} метод применяется к интерпретации ГДИ и обсуждаются вопросы суммирования рядов.
	Распространение метода на случай многофазной фильтрации можно найти в \cite{posv2}.
	В радиальном случае проблема суммирования ряда, отмеченная в \cite{charniy}, до сих пор не решена.
	Решение задач методом функций Грина в неоднородных областях преставляет большие трудности.

	Зачастую, как в аналитических, так и в численных рассмотрениях пренебрегается зависимостью PVT характеристик континуумов от температуры (расщепление по физическим процессам). Несмотря на это, уравнение баланса энергии в случае многофазной фильтрации имеет сложный вид, его решение в общем виде неизвестно.
	В случае однофазной фильтрации и в пренебрежение теплопроводностью уравнение имеет гиперболический вид и имеет аналитическое решение \cite{checkalyuk,ramazanov_spe}.
	В работе \cite{duru} задача решается аналитически посредством расщепления на гиперболическое (конвекция, фазовые переходы, эф-т Джоуля-Томпсона, адиабатика) и параболическое (теплопроводность) уравнения.

	Основным подходом при решении рассматриваемых задач является использование численных методов решения уравнений в частных производных \cite{petrov}.
	Численные методы, схемы и варианты аппроксимаций уравнений фильтрации подробно рассмотрены в \cite{kanevskaya, aziz, chen}.
	Различные разностные схемы решения многофазных постановок описаны в \cite{keldysh}.
	Используемый в данной работе, метод конечных объёмов, подробно описан в \cite{leveque}.
	
	Как уже было отмечено, структура ОЗП сложна. Подробное изложение воздействия различных технологических и геологических процессов на пласт, в целом, и на ОЗП, в частности, представлено в \cite{civan}.
	Исследованию процессов формирования и разрушения структуры ОЗП, их математическому моделированию посвящена работа \cite{makarova}.
	Корреляции для геометрического скина, отвечающего за перфорационные каналы, представлены в \cite{tariq} и основаны также на численных расчётах.
	
	Классическим пособием по ГДИ является книга Р. Эрлагера \cite{erlauger}.
	Современная техника интерпретации ГДИ основана на аналитическом решении уравнения пьезопроводности с использованием различных интегральных преобразований (Фурье, Лапласа и его модификаций). Подробности можно найти в \cite{kappa}.
	Основным результатом классического исследования ГДИ является оценка производительности скважины. Приближённые методы определения коэффициэнта продуктивности, скин-фактора, основы узлового анализа (nodal analysis) скважин описаны в \cite{mukerdzhi}.
	

	Метод \textit{термозондирования} пласта был впервые представлен Э.Б. Чекалюком и назван его именем.
	Он представлен в \cite{checkalyuk} наряду с другими приложениями термометрии и тепловыми методами воздействия на пласт.
	Метод прост и легок в использовании. Недостатком является то, что он учитывает лишь эф-т Джоуля-Томпсона.
	Модели неизотермической фильтрации подробно описываются в \cite{alishaev}, где основной упор сделан на исследование методов термозаводнения.
	Дальнейшее развитие ТГДИ связано с А.Ш. Рамазановым \cite{ramazanov_diss}, Р.А. Валиуллиным \cite{valiullin}.
	В работе \cite{ramazanov_old1} эф-т Джоуля-Томпсона исследуется экспериментально.
	Непосредственное использование нестационарных термограмм в комплексе с ГДИ обсуждается в \cite{ramazanov_old2}.
	
	Современные исследования тесно связаны с использованием различных численных симуляторов, позволяющих применять ТГДИ для достаточно сложных моделей многофазной фильтрации. В статье \cite{ramazanov_spe} представлены примеры интерпретации промысловых данных ТГДИ и получены характеристики ОЗП, также рассмотрены некоторые простейшие модели учёта ствола скважины. 
	В работе \cite{ramazanov_main} рассмотрено аналитическое решение для кусочно-однородной области (ОЗП и пласт) с учётом разгазирования нефти.
	Сопряжённая модель скважина-пласт-горные породы представлена в \cite{ramazanov_spe1}, где, в комбинации с другими исследованиями, применяется ТГДИ для определения профилей притока и ФЕС каждого из интервалов перфорации.
	
\subsection*{Актуальность работы}
	Представленная работа является актуальной и имеет непосредственное приложение в области промысловых и геофизических исследований скважин.

	На текущий момент ГДИ входят в перечень обязательных исследований скважин как на этапе освоения, так и при последующей эксплуатации.
	Результатом этих исследований, кроме всего прочего, являются промысловые данные по температуре с забоя скважины, измеренные с хорошей точностью.
	В настоящий момент эти данные практически ни как не используются, либо используются не достаточно эфективно при интерпретации ГДИ.
	В то время как, в работах \cite{checkalyuk, ramazanov_old2, valiullin, ramazanov_diss, ramazanov_spe, ramazanov_spe1, posv} представлена методика интепретации этих данных, позволяющая получить дополнительную информацию о ОЗП и профилях притока многопластовой системы, которую не может быть найдена в результате ГДИ. Метод имеет условное название ТГДИ.
	Эта информация позволит лучше понимать структуру ОЗП, позволит повысить достоверность и, как следствие, эффективность планируемых методов увеличения нефтеотдачи (МУН).
	
	К тому же, в последнее время резко возрос интерес к третичным МУН, основу которых составляют тепловые методы поддержания энергии пласта, изменения его физико-химических свойств.
	Такая тенденция напрямую связана с проблемами разработки так называемых трудноизвлекаемых месторождений углеводородов.
	Очевидно, что при моделировании фильтрации и, непосредственно, МУН на таких месторождениях за основу должна уже браться модель неизотермической фильтрации, учитывающая различные физико-химические процессы происходящие в пласте.
	В таких условиях, роль, представленных в данной работе исследований ОЗП, возростает.
	
\subsection*{Научная новизна}
	В отличие от упомянутых выше работ, разработанные в данной статье термогидродинамические модели учитывают совместное влияние на динамику поля температуры и давления в пласте $T\left(t, r\right)$, $P\left(t,r\right)$ следующих параметров, процессов и эффектов: разнонаправленный эффект Джоуля-Томсона при совместной фильтрации различных флюидов, адиабатическое расширение, фазовый переход флюидов в пласте при работе на давлении ниже давления насыщения, наличие зоны поражения пласта с измененными ФЕС, вторичное вскрытие пласта системой перфорационных каналов. Проведенное в работе исследование позволяет определить информативность данных динамики забойной температуры для определения параметров околоскважинной зоны в более интересном с практической точки зрения случае – многофазная фильтрация к скважине с вторичным вскрытием.

\subsection*{Апробация}
	Данная работа была представлена на следующих конференциях:
\begin{itemize}
	\item 58-ая научная конференция Московского физико-технического института, Долгопрудный, 23-28 ноября,  2015 г.
	\item Математическое моделирование и компьютерные технологии в процессах разработки месторождений, Уфа, 7-9 апреля 2015 г.
\end{itemize}

	Представленная работа отобрана для технической сессии конференции Общества инженеров нефтегазовой промышленности (SPE):
\begin{itemize}
	\item Гайдуков Л.А., Новиков А.В., Посвянский Д.В. Исследование термогидродинамических процессов при многофазной фильтрации флюидов к скважине в техногенно-измененном пласте со вторичным вскрытием с целью определения параметров околоскважинной зоны. SPE-181964 // Российская нефтегазовая техническая конференция и выставка SPE. 24-26 октября 2016. Москва.
\end{itemize}
	
