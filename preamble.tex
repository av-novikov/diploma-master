\section*{Введение}
\addcontentsline{toc}{section}{Введение}
\setcounter{subsection}{0}
	
	Традиционно расчёт фильтрации флюидов в нефтегазоносных пластах проводят пренебрегая изменением температуры среды. Во-первых, потому, что изменения температуры при фильтрации крайне малы, и не оказывают сколько-нибудь заметного влияния на свойства среды. Во-вторых, для записи температурных данных необходимы высокоточные датчики, которые получили широкое распространение лишь в последнее десятилетие. В-третьих, при дополнительном расчёте теплопереноса возникает необходимость указания большого количества характеристик (как компонентов среды, так и задачи в целом), значения которых неизвестны, либо известны с весьма плохой точностью. В этой связи, модели неизотермической фильтрации при гидродинамическом моделировании месторождений остаются не у дел.

	Тем не менее существует ряд задач, в которых привлечение термодинамики насыщенной пористой среды позволяет определить некоторые \textit{эксплуатационные характеристики пласта}. Речь идет о измерении \textit{давления}, \textit{дебита} и \textit{температуры} в стволе скважины и последующей совместной интерпретации этих данных на основе соответствующих моделей массо- и теплопереноса.
Такие исследования носят название \textit{термогидродинамических}, и подразделяются на два вида:
\begin{itemize}
	\item измерение параметров вдоль ствола скважины;
	\item измерение параметров во времени на определённой глубине.
\end{itemize}
	
	Первый тип исследования относится к стандартным \textit{геофизическим} исследованиям, проводимым на этапе освоения и эксплуатации скважин. Их результатом является определение следующих хакрактеристик пласта:
\begin{itemize}
	\item выявление работающих пластов;
	\item определение интервалов притока;
	\item оценка состава флюида, определение интервалов притока воды, нефти и газа;
	\item определение дебита, обводнённости, коэффициента продуктивности;
	\item определение пластового давления и температуры.
\end{itemize}
	
	Результаты исследований по второму типу представляют особую ценность, т.к. позволяют определить динамические характеристики экплуатируемых пластов. 
	Широко распространённой техникой являются, так называемые, \textit{гидродинамические исследования} (ГДИ). 
	Этот метод позволяет, на основе интерпретации кривых востановления уровня (КВУ), кривых востановления давления (КВД), определить проницаемость пласта, суммарный  скин-фактор, продуктивность скважины, которая, в свою очередь, является основным показателем работы скважины.
	Тем не менее, метод не даёт никаких более подробных сведений о структуре околоскважинной зоны пласта (ОЗП), тогда как с гидродинамической точки зрения эта область представляет наибольший интерес.

	Начиная с первичного вскрытия пласта, когда происходит проникновение фильтрата бурового раствора, формируется сложная структура ОЗП, в результате чего, меняются и её фильтрационно-ёмкостные свойства (ФЕС). На этапе заканчивания, пласт прошивается системой перфорационных каналов, которая существенно меняет картину теченения в ОЗП. Эти и другие технологические процедуры серьёзным образом усложняют моделирование многофазного притока к скважине.
	
	Для определения структуры ОЗП, в последнее время начинают применяться \textit{термогидродинамические исследования} (ТГДИ) скважин \cite{ramazanov_diss,ramazanov_spe, ramazanov_spe1, posv}, основа которых была заложена ещё более полувека тому назад Э.Б. Чекалюком \cite{checkalyuk}.
	Эти исследования основаны на моделях неизотермической фильтрации, с помощью которых расчитывается забойная температура.
	Они используются для интерпретации промысловых данных, дополняя развитую технику интерпретации ГДИ.

	Поскольку рассматриваемые исследования основаны на интерпретации нестационарных данных, принципиальным моментом здесь является скорость процесса.
	Давление очень быстро реагирует на изменение режима добычи и не позволяет получить информацию о ОЗП.
	В свою очередь, скорость основных тепловых процессов существенно ниже, что даёт возможность использовать ТГДИ для исследования ОЗП.

	На пути прогресса распространения этой методики долгое время стояла недостаточная разрешающая способность темпертурных датчиков.
	Современные же устройства позволяют определять температуру с точностью до 0,0001 K, что является достаточным для идентификации даже незначительных изменений температурного фона в стволе скважины.

	Тем не менее стоит отметить, что непосредственное применение ТГДИ на практике возможно лишь при наличии соотвествующего оборудования и его правильной установки, проведении исследований тепловых PVT свойств флюида и породы, определённых системах заканчивания скважины, вкупе с другими геофизическими и промысловыми исследованиями.

\subsection*{Обзор литературы}
	Подробный вывод и описание законов сохранения и определяющих соотношений механики и термодинамики насыщенной пористой среды можно найти в \cite{kondaurov}.
	Описание процессов тепло- и массо- переноса в многофазных многокомпонентных средах представлено в \cite{nigmatulin, multiphase}. 
	Основные задачи фильтрации флюидов в нефтегазоносных пластах рассмотрены в \cite{basniev, charniy}.
	Модели неизотермической фильтрации представлены в \cite{checkalyuk, alishaev}.

	Математической моделью задач тепло- и массопереноса в пористых средах, как правило, является смешанная задача для системы нелинейных уравнений в частных производных. Для однородной прямоугольной области области задача массопереноса имеет аналитическое решение через функцию Грина \cite{vladimirov}.
	Применение функций Грина для расчёта притока к скважинам произвольной геометрии рассмотрено в \cite{aziz_green}.
	В \cite{posv1} метод применяется к интерпретации ГДИ и обсуждаются вопросы суммирования рядов.
	Распространение метода на случай многофазной фильтрации можно найти в \cite{posv2}.
	В радиальном случае проблема суммирования ряда, отмеченная в \cite{charniy}, до сих пор не решена.
	Решение задач методом функций Грина в неоднородных областях преставляет большие трудности.

	Зачастую, как в аналитических, так и в численных рассмотрениях пренебрегается зависимостью PVT характеристик континуумов от температуры (расщепление по физическим процессам). Несмотря на это, уравнение баланса энергии в случае многофазной фильтрации имеет сложный вид, его решение в общем виде неизвестно.
	В случае однофазной фильтрации и в пренебрежение теплопроводностью уравнение имеет гиперболический вид и имеет аналитическое решение \cite{checkalyuk,ramazanov_spe}.
	В работе \cite{duru} задача решается аналитически посредством расщепления на гиперболическое (конвекция, фазовые переходы, эф-т Джоуля-Томпсона, адиабатика) и параболическое (теплопроводность) уравнения.

	Основным подходом при решении рассматриваемых задач является использование численных методов решения уравнений в частных производных \cite{petrov}.
	Численные методы, схемы и варианты аппроксимаций уравнений фильтрации подробно рассмотрены в \cite{kanevskaya, aziz, chen}.
	Используемый в данной работе, метод конечных объёмов, подробно описан в \cite{leveque}.
	Различные разностные схемы решения многофазных постановок описаны в \cite{keldysh}.
	
\subsection*{Актуальность}

\subsection*{Новизна}

\subsection*{Апробация}