\section*{Глава 2. Неизотермическая фильтрация флюида к несовершенным скважинам.}
\addcontentsline{toc}{section}{Глава 2. Неизотермическая фильтрация флюида к несовершенным скважинам.}

\setcounter{section}{2}
\setcounter{subsection}{0}
\setcounter{equation}{0}


\subsection{Виды несовершенств}
	В ходе бурения, освоения и эксплуатации скважины различные технологические процессы воздействуют на ОЗП.
	В результате её ФЕС меняются.
	Отличие притока скважины от притока, рассчитанного по формуле Дюпюи \eqref{dupuit} (при $s=0$) учитывают, вводя \textit{скин-фактор} $s$:
\begin{equation}
	\label{dupuit}
	Q = \frac{2\pi k h}{\mu B \left(\ln\displaystyle\frac{r_e}{r_w}+s\right)}
\end{equation}

	В процессе бурения скважины происходит загрязнение ОЗП фильтратом бурового раствора