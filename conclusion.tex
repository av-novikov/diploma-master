\section*{Выводы}
\addcontentsline{toc}{section}{Выводы}
\setcounter{subsection}{0}
	Подводя итог, можно выделить следующие результаты данной работы:
\begin{enumerate}
	\item{Был написан расчётный модуль моделирования неизотермической фильтрации в пласте в простейшем случае однофазной фильтрации. Модуль был верифицирован на имеющейся аналитике. был проведён анализ чувствительности модели к параметрам различных эфектов, составляющих температурную аномалию на забое (эф-т Джоуля-Томпсона, адиабатическое расширение, конвективный и кондуктивный перенос тепла), в некоторых постановка выделены основные эффекты, показаны пределы их доминирования.}
	\item{Модуль был расширен для учёта ухудшенных ФЕС ОЗП. Было показано что динамика забойной температуры чувствительна к параметрам ОЗП (размер, проницаемость), тогда как давление -- не чувствительно, проведено исследования влияния зарактеристик ОЗП на динамику температуры.}
	\item{Модуль был расширен для учёта вторичного вскрытия -- перфорационных каналов. Была проведена верификация полученной модели на соответствующей аналитике относительно давления. Проведено исследование влияния данного несовершенства скважины на динамику забойной температуры.}
	\item{Проведено исследование совместного влияния несовершенств скважины (ухудшенные ФЕС ОЗП, перфорационные каналы) на динамику забойной температуры. Показано, что для случая, когда каналы прошивают ОЗП, показания температуры не информативны относительно свойств ОЗП.}
	\item{Модуль был расширен для расчёта двухфазной двухкомпонентной неизотермической фильтрации с фазовыми переходами. Было проведён анализ чувствительности модели к теплоте фазового перехода, указаны случаи, когда данный эффект необходимо брать во внимание.}
	\item{Написанный модуль трёхмерного расчёта двухфазной двухкомпонентной неизотермической фильтрации был применён для интерпретации реальных скважинных данных для месторождений Западной Сибири. Были расчитаны характеристики ОЗП соответствующих скважин.}
	\item{Для расчёта некоторых гидродинамических постановок были использованы полуаналитические методы, представленные в работе. Метод функций Грина с алгоритмом суммирования Эвальда показал отличные характеристики производительности, в сравнении с численными аналогами. Была показана возможность применения метода для расчёта притока многофазой многокомпонентной жидкости к скважине произвольной геометрии.}
\end{enumerate}

	Основные результаты данной работы были (будут) представлены на следующих конференциях:
\begin{itemize}
	\item{Гайдуков Л.А., Новиков А.В., Посвянский Д.В., Тухватуллина Р.Р. Математическое моделирование термогидродинамических процессов в пласте для определения структуры околоскважинной зоны. // Математическое моделирование и компьютерные технологии в процессах разработки месторождений, Уфа,  7-9 апреля 2015 г.}
	\item{Гайдуков Л.А., Новиков А.В., Посвянский Д.В., Тухватуллина Р.Р. Математическое моделирование термогидродинамических процессов в пласте для определения структуры околоскважинной зоны. // 58-ая научная конференция МФТИ, Долгопрудный,  23-28 ноября,  2015 г.}
	\item{Гайдуков Л.А., Новиков А.В., Посвянский Д.В. Исследование термогидродинамических процессов при многофазной фильтрации флюидов к скважине в техногенно-измененном пласте со вторичным вскрытием с целью определения параметров околоскважинной зоны. SPE-181964 // Российская нефтегазовая техническая конференция и выставка SPE. 24-26 октября 2016. Москва.}
\end{itemize}